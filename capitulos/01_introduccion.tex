%%%%%%%%%%%%%%%%%%%%%%%%%%%%%%%%%%%%%%%%%%%%%%%%%%%%%%%%%%%%%%%%%%%%%%%%%%%%%%%%%
% Introducción
%%%%%%%%%%%%%%%%%%%%%%%%%%%%%%%%%%%%%%%%%%%%%%%%%%%%%%%%%%%%%%%%%%%%%%%%%%%%%%%%%

\chapter{Introducción} % (fold)
\label{cha:Introduccion}

    Para empezar se decide entre hacer detección binaria de la entrada, es decir, si el parche ha sido golpeado o no, y
    hacer que estas entradas sean concurrentes (al golpear dos parches, el sonido de ambos suena al mismo tiempo), o
    hacer detección de distintos sonidos en un mismo parche, dependiendo de cómo se golpee el parche (en el centro, en
    el lateral, con más o menos fuerza…) el sonido emitido es diferente.\newline

    Se decide empezar con la primera alternativa y dejar la segunda para más adelante en caso de tener tiempo.

    \section{Derechos de autor} % (fold)
    \label{sec:DerechosDeAutor}
        \subsection{¿Qué son los derechos de autor?} % (fold)
        \label{sub:QueSonLosDerechosDeAutor}

            Los derechos de autor son una serie de leyes que protegen la autoría de las obras. Estas pueden ser libros,
            películas, obras de teatro, programas informáticos...\newline

            Se cubren dos tipos de derechos: los derechos patrimoniales, que aseguran que el autor obtenga compensación
            financiera, y los derechos morales, que cubren todo lo que no esté relacionado con los derechos
            patrimoniales, por ejemplo, la prohibición de que se modifique la obra.\cite{derechos_ompi}

        % section ¿Qué son los derechos de autor? (end)

        \subsection{Historia de los derechos de autor} % (fold)
        \label{sub:HistoriaDeLosDerechosDeAutor}

            La historia de los derechos de autor comienza en 1710, cuando se publica el Estatuto de la Reina
            Anna\cite{estatuto_anna} que fue el primer reglamento sobre los derechos de autor. En el momento de la
            publicación de éste estatuto solo se contemplaban los derechos sobre los libros, pero en posteriores leyes
            se contemplan otros usos, como cine, radio, fotografías o programas de ordenador.\newline

            En la actualidad, los derechos de autor se protegen tanto con acuerdos y leyes internacionales, como leyes
            nacionales.\newline

            Desde 1974 en Estados Unidos con la CONTU\cite{contu} (Commission on New Technological Uses of Copyrighted
            Works) y de 1991 en la Unión Europea con la Computer Programs Directive\cite{com_pro_dir}, se protegen los
            derechos de autor de los programas informáticos.

        % section Historia de los derechos de autor (end)

        \subsection{Origen de sonidos de batería} % (fold)
        \label{sub:OrigenDeSonidosDeBateria}

            Los sonidos de batería han sido obtenidos de Soundsnap\cite{soundsnap} y de la biblioteca de sonidos de
            GarageBand para macOS\cite{garageband}

        % section Origen de sonidos de batería (end)

    \section{Planificación} % (fold)
    \label{sec:Planificacion}

        \subsection{A priori optimista} % (fold)
        \label{sub:APrioriOptimista}

            %% TODO: Mejorar predicción de tiempos a priori
            \begin{itemize}
                \item
                    Programa que genere los sonidos de batería concurrentes: 1-2 meses
                \item
                    Construcción de un prototipo de batería con sonidos concurrentes: 1-2 meses
                \item
                    Añadir funcionalidad que genere diferentes sonidos de un mismo parche/platillo: 1 mes
                %% El programa tiene que diferenciar el lugar en el que se golpea el parche/platillo,
                %% lo que será la mayor dificultad
                \item
                    Añadir sensores y conexiones para generar diferentes sonidos según dónde se golpee en el
                    parche/platillo: 1-2 semanas
                \item
                    Aplicación web para conectar dos baterías online: 1-2 meses
                \item
                    Documentación: 3 semanas
            \end{itemize}

        % subsection A priori optimista (end)

        \subsection{A posteriori (en caso de haber diferencias)} % (fold)
        \label{sub:APosterioriEnCasoDeHaberDiferencias)}

        % subsection A posteriori (en caso de haber diferencias) (end)

    % section Planificación (end)

% chapter Introducción (end)