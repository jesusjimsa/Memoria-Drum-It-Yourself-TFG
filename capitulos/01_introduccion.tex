%%%%%%%%%%%%%%%%%%%%%%%%%%%%%%%%%%%%%%%%%%%%%%%%%%%%%%%%%%%%%%%%%%%%%%%%%%%%%%%%%
% Introducción
%%%%%%%%%%%%%%%%%%%%%%%%%%%%%%%%%%%%%%%%%%%%%%%%%%%%%%%%%%%%%%%%%%%%%%%%%%%%%%%%%

\chapter{Introducción} % (fold)
\label{cha:Introduccion}

    \section{Historia de los instrumentos digitales} % (fold)
    \label{sec:HistoriaDeLosInstrumentosDigitales}

        Los primeros intentos de guardar y reproducir el sonido fueron analógicos. Estos métodos captan las ondas y las
        almacenan en diferentes medios para su posterior reproducción, como en un disco de vinilo o un casete
        \cite{historia_instrumentos_digitales}.

        Con la aparición de la informática y los ordenadores se empiezan a almacenar estos sonidos en un formato
        digital, capaz de ser interpretado por ordenadores. En este caso, los sonidos se guardan en forma de bytes en
        diferentes formatos de archivo, como MP3, AAC, OGG... Al igual que el software, los formatos de audio pueden ser
        abiertos o cerrados. Por ejemplo, WMA \cite{wikipedia_wma} o AAC \cite{wikipedia_aac} son formatos propietarios,
        mientras que OGG \cite{wikipedia_ogg} o ALAC \cite{wikipedia_alac} son formatos abiertos. MP3 pasó a ser un
        formato abierto a partir de 2017 \cite{mp3_licencia}.

        Los formatos de archivo de audio se pueden ordenar en tres categorías principales:

        \begin{table}[ht]
            \resizebox{\textwidth}{!}{%
            \begin{tabular}{|l|l|l|}
                \hline
                \textbf{No comprimidos} & \textbf{Compresión con pérdidas} & \textbf{Compresión sin pérdidas} \\ \hline
                AU, WAV, AIFF...        & MP3, AAC, OGG...                 & FLAC, ALAC, WMA...               \\ \hline
            \end{tabular}%
            }
            \caption{Tabla con ejemplos de formatos de las tres categorías \cite{formatos_audio}}
            \label{tab:my-table}
        \end{table}

        \begin{itemize}
            \item \textbf{No comprimidos}: Consisten en capturar las ondas del sonido y guardarlas en archivos sin
            ningún procesamiento posterior.
            \item \textbf{Compresión con pérdidas}: En el proceso de compresión se pierde algo de información y, con
            ella, algo de calidad. A cambio de la pérdida de calidad, se obtienen ficheros más ligeros.
            \item \textbf{Compresión sin pérdidas}: Al contrario que en la compresión con pérdidas, en este tipo de
            compresión no se pierde nada de información ni de calidad de audio, sin embargo, se obtienen archivos más
            pesados que en la categoría anterior.
        \end{itemize}

        Los primeros intentos de instrumento musical no analógico se pueden encontrar en los sintetizadores. Estos
        instrumentos utilizan la electricidad para producir las ondas del sonidos pasándola por una serie de módulos.
        Hasta la década de 1980, cada fabricante utilizaba su propio estándar para la sincronización de los sonidos de
        los sintetizadores. En 1981, la empresa Oberheim Electronics comenzó a contactar con otros fabricantes para
        desarrollar un estándar, de ese modo apareció MIDI. Este estándar describe el protocolo de comunicación, la
        interfaz digital y las conexiones electrónicas que deben llevar los diferentes tipos de dispositivos
        electrónicos para reproducir, editar y grabar música \cite{midi_wikipedia}.

        En cuanto a instrumentos musicales electrónicos podemos encontrar desde un teclado o una guitarra a una batería.

        En el área de instrumentos digitales nos encontramos con los instrumentos VST (Virtual Studio Technology). Estos
        instrumentos toman muestras de sonidos de diferentes instrumentos y, mediante un teclado y un ordenador,
        programar estos sonidos y componer y grabar cualquier tipo de canción \cite{historia_instrumentos_digitales}.

        \begin{figure}[ht]
            \centering
            \includegraphics[width=\textwidth/2]{ejemplo_sampler}
            \caption{Ejemplo de sampler digital de la marca AKAI Pro \cite{akai_pro_imagen}\label{fig:EjemploSampler}}
        \end{figure}

    % section Historia de los instrumentos digitales (end)

    \section{Partes de una batería} % (fold)
    \label{sec:PartesDeUnaBateria}

        A continuación se explicarán las partes principales de una batería. Se ha elegido el instrumento musical de la
        batería porque, personalmente, es un instrumento que me gusta y toco desde hace unos años. Tengo una batería
        eléctrica y me parecía interesante hacer una por mí mismo.

        Las partes de una batería son las siguientes:

        \begin{itemize}
            \item \textbf{Caja}: Su función principal suele ser la de marcar los compases.
            \item \textbf{Toms}: Son los tambores más numerosos en una batería.
            \item \textbf{Bombo}: Se toca con un pedal y produce el sonido más grave de la batería. Se utiliza para
            llevar la base del ritmo.
            \item \textbf{Platillo crash}: Se utiliza para dar énfasis y suele ir acompañado del bombo.
            \item \textbf{Platillo hi-hat}: Consta de dos platillos que se pueden abrir o cerrar con un pedal y se
            utiliza para llevar el ritmo de la canción.
            \item \textbf{Platillo ride}: Puede usarse para llevar el ritmo en lugar de con el hi-hat.
        \end{itemize}

        \begin{figure}[ht]
            \centering
            \includegraphics[width=10cm]{partes_bateria}
            \caption{Partes de una batería \cite{partes_bateria_fuente}\label{fig:PartesBateria}}
        \end{figure}

    % section Partes de una batería (end)

    \section{Planificación del trabajo} % (fold)
    \label{sec:PlanificacionDelTrabajo}

        \subsection{Etapas de desarrollo del trabajo} % (fold)
        \label{sub:EtapasDeDesarrolloDelTrabajo}

            \begin{itemize}
                \item \textbf{1ª etapa:} Estudio del problema.

                \item \textbf{2ª etapa:} Búsqueda de bibliotecas de reproducción de sonido.

                \item \textbf{3ª etapa:} Implementación del software.

                \item \textbf{4ª etapa:} Construcción de la batería.

                \item \textbf{5ª etapa:} Documentación.
            \end{itemize}

        % subsection Etapas de desarrollo del trabajo (end)

        \subsection{A priori optimista} % (fold)
        \label{sub:APrioriOptimista}

            En el siguiente diagrama de Gantt se detalla una planificación de cómo se espera desarrollar el proyecto en
            el tiempo.

            De los nueve meses que se dedicarán al proyecto, seis estarían enfocados en el desarrollo del
            software que controlará la batería, dejando unos tres meses a la creación de prototipos y versiones
            secuenciales del programa, y otros tres meses al desarrollo de la versión concurrente y final del software.

            Los meses restantes, se dedicarían a la construcción de la batería y la escritura de la memoria, dejando el
            mes de mayo libre, en caso de que hubiera imprevistos y alguna de las fases se alargara.

            \begin{figure}[ht]
                \centering
                \includegraphics[width=\textwidth]{planificacion_gantt}
                \caption{Diagrama de Gantt de la planificación por etapas\label{fig:PlanificacionGantt}}
            \end{figure}

        % subsection A priori optimista (end)

        \subsection{A posteriori} % (fold)
        \label{sub:APosteriori}

            Una vez terminado el proyecto, se puede ver en las gráficas que proporciona Github sobre cuándo se han
            realizado los commits en los dos repositorios (código y memoria) cuál es la diferencia entre la
            planificación a priori y lo que ha terminado resultando.

            \begin{figure}[ht]
                \centering
                \includegraphics[width=\textwidth]{commits_codigo}
                \caption{Gráfica de los commits en el repositorio del código\label{fig:CommitsCodigo}}
            \end{figure}

            \begin{figure}[ht]
                \centering
                \includegraphics[width=\textwidth]{commits_memoria}
                \caption{Gráfica de los commits en el repositorio de la memoria\label{fig:CommitsMemoria}}
            \end{figure}

            Se puede ver en las imágenes que, además de haber empezado en agosto, en lugar de en septiembre, con el
            código, éste ha seguido bastante la planificación previamente indicada. Si nos vamos al historial de commits
            del repositorio, se puede observar también que el desarrollo del sistema de sonido concurrente también sigue
            a la planificación a priori bastante bien.

            Es en la escritura de la memoria en la que se ven más diferencias, ya que toma gran parte de marzo y mayo,
            meses que no estaban planificados para la memoria. Esto se debe, mayormente a subestimar la dificultad del
            desarrollo de la misma y a poca experiencia en la escritura de este tipo de documentos.

            La construcción de la batería se fue realizando a lo largo de los diez meses que he dedicado a este
            proyecto.

        % subsection A posteriori (end)

    % section Planificación del trabajo (end)

% chapter Introducción (end)
