%%%%%%%%%%%%%%%%%%%%%%%%%%%%%%%%%%%%%%%%%%%%%%%%%%%%%%%%%%%%%%%%%%%%%%%%%%%%%%%%%
% Introducción
%%%%%%%%%%%%%%%%%%%%%%%%%%%%%%%%%%%%%%%%%%%%%%%%%%%%%%%%%%%%%%%%%%%%%%%%%%%%%%%%%

\chapter{Introducción} % (fold)
\label{cha:Introduccion}

    Para empezar se decide entre hacer detección binaria de la entrada, es decir, si el parche ha sido golpeado o no, y
    hacer que estas entradas sean concurrentes (al golpear dos parches, el sonido de ambos suena al mismo tiempo), o
    hacer detección de distintos sonidos en un mismo parche, dependiendo de cómo se golpee el parche (en el centro, en
    el lateral, con más o menos fuerza…) el sonido emitido es diferente.\newline

    Se decide empezar con la primera alternativa y dejar la segunda para más adelante en caso de tener tiempo.

    \section{Historia de los sintetizadores} % (fold)
    \label{sec:HistoriaDeLosSintetizadores}

    %%%%%%%%%%%%%%%% Primeros instrumentos con ordenadores (¿guitarra?)
    %%%%%%%%%%%%%%%% Instrumentos digitales

    % section Historia de los sintetizadores (end)

    \section{Planificación} % (fold)
    \label{sec:Planificacion}

        \section{Etapas} % (fold)
        \label{sec:Etapas}

            \begin{itemize}
                \item \textbf{1ª etapa:} Estudio del problema.

                \item \textbf{2ª etapa:} Búsqueda de librerías de reproducción de sonido.

                \item \textbf{3ª etapa:} Implementación del software.

                \item \textbf{4 a etapa:} Construcción de la batería.

                \item \textbf{5ª etapa:} Documentación.
            \end{itemize}
            

        % section Etapas (end)

        \subsection{A priori optimista} % (fold)
        \label{sub:APrioriOptimista}

            %% TODO: Mejorar predicción de tiempos a priori
            \begin{itemize}
                \item
                    Programa que genere los sonidos de batería concurrentes: 1-2 meses
                \item
                    Construcción de un prototipo de batería con sonidos concurrentes: 1-2 meses
                \item
                    Añadir funcionalidad que genere diferentes sonidos de un mismo parche/platillo: 1 mes
                %% El programa tiene que diferenciar el lugar en el que se golpea el parche/platillo,
                %% lo que será la mayor dificultad
                \item
                    Añadir sensores y conexiones para generar diferentes sonidos según dónde se golpee en el
                    parche/platillo: 1-2 semanas
                \item
                    Aplicación web para conectar dos baterías online: 1-2 meses
                \item
                    Documentación: 3 semanas
            \end{itemize}

        % subsection A priori optimista (end)

        \subsection{A posteriori (en caso de haber diferencias)} % (fold)
        \label{sub:APosterioriEnCasoDeHaberDiferencias)}

        % subsection A posteriori (en caso de haber diferencias) (end)

    % section Planificación (end)

% chapter Introducción (end)