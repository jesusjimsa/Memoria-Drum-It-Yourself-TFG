%%%%%%%%%%%%%%%%%%%%%%%%%%%%%%%%%%%%%%%%%%%%%%%%%%%%%%%%%%%%%%%%%%%%%%%%%%%%%%%%%
% Aspectos legales
%%%%%%%%%%%%%%%%%%%%%%%%%%%%%%%%%%%%%%%%%%%%%%%%%%%%%%%%%%%%%%%%%%%%%%%%%%%%%%%%%

\chapter{Aspectos legales} % (fold)
\label{cha:AspectosLegales}

    \section{Derechos de autor} % (fold)
    \label{sec:DerechosDeAutor}
        \subsection{¿Qué son los derechos de autor?} % (fold)
        \label{sub:QueSonLosDerechosDeAutor}

            Los derechos de autor son una serie de leyes que protegen la autoría de las obras. Estas pueden ser libros,
            películas, obras de teatro, programas informáticos...

            Se cubren dos tipos de derechos: los derechos patrimoniales, que aseguran que el autor obtenga compensación
            financiera, y los derechos morales, que cubren todo lo que no esté relacionado con los derechos
            patrimoniales, por ejemplo, la prohibición de que se modifique la obra \cite{derechos_ompi}.

        % section ¿Qué son los derechos de autor? (end)

        \subsection{Historia de los derechos de autor} % (fold)
        \label{sub:HistoriaDeLosDerechosDeAutor}

            La historia de los derechos de autor comienza en 1710, cuando se publica el Estatuto de la Reina
            Anna \cite{estatuto_anna} que fue el primer reglamento sobre los derechos de autor. En el momento de la
            publicación de este estatuto solo se contemplaban los derechos sobre los libros, pero en posteriores leyes
            se contemplan otros usos, como cine, radio, fotografías o programas de ordenador.

            En la actualidad, los derechos de autor se protegen tanto con acuerdos y leyes internacionales, como leyes
            nacionales.

            Desde 1974 en Estados Unidos con la CONTU \cite{contu} (Commission on New Technological Uses of Copyrighted
            Works) y de 1991 en la Unión Europea con la Computer Programs Directive \cite{com_pro_dir}, se protegen los
            derechos de autor de los programas informáticos.

        % subsection Historia de los derechos de autor (end)

        \subsection{Derechos de autor en España} % (fold)
        \label{sub:DerechosDeAutorEnEspana}

            En España tenemos la Ley de Propiedad Intelectual. En esta ley, la propiedad intelectual se define como:

            ``La obra literaria, artística o científica, expresadas en cualquier medio (libros, escritos, composiciones
            musicales, obras, coreografías, obras audiovisuales, esculturas, obras pictóricas, planos, maquetas, mapas,
            fotografías, programas de ordenador y bases de datos) que corresponde a su autor por el solo hecho de su
            creación, que tiene derecho a explotarla y disponer de ella a su voluntad.''

            \begin{flushright}
                (La Ley de Propiedad Intelectual y los derechos de autor en España. \cite{propiedad_intelectual_espana})
            \end{flushright}

            \subsubsection{Tipos de derechos} % (fold)
            \label{ssub:TiposDeDerechos}

                En la Ley de Propiedad Intelectual se definen dos grupos de derechos de autor:

                \begin{itemize}
                    \item \textbf{Derechos morales}: Corresponden al autor de la obra y no se puede renunciar a ellos ni
                    traspasarlos. Se encargan de proteger la identidad y reputación del autor.
                    \item \textbf{Derechos patrimoniales}: Estos sí se pueden traspasar. Sirven para que el autor decida
                    cómo se utiliza y representa su obra. No se puede usar la obra sin autorización expresa.
                \end{itemize}

            % subsubsection Tipos de derechos (end)

            \subsubsection{Programas de ordenador} % (fold)
            \label{ssub:ProgramasDeOrdenador}

                La Ley de Propiedad Intelectual define \textit{programa de ordenador} como:

                ``A los efectos de la presente Ley se entenderá por programa de ordenador toda secuencia de
                instrucciones o indicaciones destinadas a ser utilizadas, directa o indirectamente, en un sistema
                informático para realizar una función o una tarea o para obtener un resultado determinado, cualquiera
                que fuere su forma de expresión y fijación.

                A los mismos efectos, la expresión programas de ordenador comprenderá también su documentación
                preparatoria. La documentación técnica y los manuales de uso de un programa gozarán de la misma
                protección que este Título dispensa a los programas de ordenador.''

                \begin{flushright}
                    (Real Decreto Legislativo 1/1996, de 12 de abril. \cite{prop_intelectual})
                \end{flushright}

                El autor del programa será la persona o grupo de personas que lo hayan creado y tendrán derechos de
                explotación sobre el programa durante la vida del autor y setenta años después de su muerte.

                El autor tendrá derecho a realizar o autorizar:

                \begin{itemize}
                    \item Reproducicción total o parcial.
                    \item Traducción, adaptación, arreglo o cualquier cambio.
                    \item Distribución pública.
                \end{itemize}

            % subsubsection Programas de ordenador (end)

        % subsection Derechos de autor en España (end)

    % section Derechos de autor (end)

    \section{Origen de sonidos de batería} % (fold)
    \label{sec:OrigenDeSonidosDeBateria}

        Los sonidos de batería han sido obtenidos de la biblioteca de sonidos de
        \href{https://www.apple.com/es/mac/garageband/}{GarageBand} para macOS. Estos sonidos son libres y gratuitos
        para usarse composiciones musicales o proyectos de audio originales, tal y como estipula la página oficial de
        Apple \cite{support_garageband}.

    % section Origen de sonidos de batería (end)

% chapter Aspectos legales (end)

\newpage
