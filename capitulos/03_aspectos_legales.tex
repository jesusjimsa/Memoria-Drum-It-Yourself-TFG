%%%%%%%%%%%%%%%%%%%%%%%%%%%%%%%%%%%%%%%%%%%%%%%%%%%%%%%%%%%%%%%%%%%%%%%%%%%%%%%%%
% Aspectos legales
%%%%%%%%%%%%%%%%%%%%%%%%%%%%%%%%%%%%%%%%%%%%%%%%%%%%%%%%%%%%%%%%%%%%%%%%%%%%%%%%%

\chapter{Aspectos legales} % (fold)
\label{cha:AspectosLegales}

    \section{Derechos de autor} % (fold)
    \label{sec:DerechosDeAutor}
        \subsection{¿Qué son los derechos de autor?} % (fold)
        \label{sub:QueSonLosDerechosDeAutor}

            Los derechos de autor son una serie de leyes que protegen la autoría de las obras. Estas pueden ser libros,
            películas, obras de teatro, programas informáticos...\newline

            Se cubren dos tipos de derechos: los derechos patrimoniales, que aseguran que el autor obtenga compensación
            financiera, y los derechos morales, que cubren todo lo que no esté relacionado con los derechos
            patrimoniales, por ejemplo, la prohibición de que se modifique la obra.\cite{derechos_ompi}

        % section ¿Qué son los derechos de autor? (end)

        \subsection{Historia de los derechos de autor} % (fold)
        \label{sub:HistoriaDeLosDerechosDeAutor}

            La historia de los derechos de autor comienza en 1710, cuando se publica el Estatuto de la Reina
            Anna\cite{estatuto_anna} que fue el primer reglamento sobre los derechos de autor. En el momento de la
            publicación de éste estatuto solo se contemplaban los derechos sobre los libros, pero en posteriores leyes
            se contemplan otros usos, como cine, radio, fotografías o programas de ordenador.

            En la actualidad, los derechos de autor se protegen tanto con acuerdos y leyes internacionales, como leyes
            nacionales.

            Desde 1974 en Estados Unidos con la CONTU\cite{contu} (Commission on New Technological Uses of Copyrighted
            Works) y de 1991 en la Unión Europea con la Computer Programs Directive\cite{com_pro_dir}, se protegen los
            derechos de autor de los programas informáticos.

        % subsection Historia de los derechos de autor (end)
    
    % section Derechos de autor (end)

    \section{Origen de sonidos de batería} % (fold)
    \label{sec:OrigenDeSonidosDeBateria}

        Los sonidos de batería han sido obtenidos de la biblioteca de sonidos de GarageBand para macOS\cite{garageband}.
        Estos sonidos son libres y gratuitos para usarse composiciones musicales o proyectos de audio originales, tal y
        como estipula la página oficial de Apple\cite{support_garageband}.

    % section Origen de sonidos de batería (end)

% chapter Aspectos legales (end)

\newpage
