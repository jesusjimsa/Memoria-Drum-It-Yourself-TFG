%%%%%%%%%%%%%%%%%%%%%%%%%%%%%%%%%%%%%%%%%%%%%%%%%%%%%%%%%%%%%%%%%%%%%%%%%%%%%%%%%
% Diseño
%%%%%%%%%%%%%%%%%%%%%%%%%%%%%%%%%%%%%%%%%%%%%%%%%%%%%%%%%%%%%%%%%%%%%%%%%%%%%%%%%

\chapter{Diseño de la propuesta} % (fold)
\label{cha:Diseno}

    En este capítulo se explican las tecnologías empleadas en el proyecto, se detallan las principales decisiones
    tomadas y se realiza una descripción a alto nivel de este trabajo.

    \section{Tecnologías empleadas} % (fold)
    \label{sec:TecnologiasEmpleadas}

        \subsection{Lenguaje de programación: C} % (fold)
        \label{sub:CLanguage}

            El lenguaje utilizado para la mayor parte del proyecto ha sido C. C es un lenguaje desarrollado en 1972 por
            Dennis Ritchie en los Laboratios Bell. Fue creado para la creación de sistemas operativos, como UNIX, y es
            muy valorado por su eficiencia.

            En la actualidad, es uno de los lenguajes más populares. Está disponible en muchas plataformas, desde
            superordenadores, hasta sistemas empotrados. Fue diseñado para mapear las instrucciones escritas en C a
            lenguaje máquina de forma muy eficiente, con pocas instrucciones, esto hace que sea muy eficiente en
            cualquier plataforma en la que se utilice \cite{wikipedia_c_language}.

            Es por esta eficiencia, por la que se ha utilizado en el proyecto. Cualquier instrumento musical, y en
            especial una batería, requiere que las acciones realizadas por el músico tengan una reacción rápida por el
            sistema, y C puede proporcionar esta velocidad.

        % subsection Lenguaje de programación: C (end)

        \subsection{Visual Studio Code} % (fold)
        \label{sub:VisualStudioCode}

            Visual Studio Code es un editor de texto diseñado por Microsoft para Windows, Linux y macOS, lanzado al
            público en 2016.

            Su sencillez, compatibilidad con una gran cantidad de lenguajes de programación y una extensa biblioteca de
            extensiones han hecho que VS Code sea uno de los editores de texto más populares entre los programadores.
            \cite{wikipedia_vs_code}

            \begin{figure}[ht]
                \centering
                \includegraphics[width=\textwidth]{vs_code}
                \caption{Captura de pantalla de Visual Studio Code \label{fig:VisualStudioCode}}
            \end{figure}

            \newpage

        % subsection Visual Studio Code (end)

        \subsection{Github} % (fold)
        \label{sub:Github}

            Github es un servicio de alojamiento de código abierto y control de versiones mediante Git.

            En este proyecto se ha utilizado principalmente para llevar ese control de versiones del código y para
            organizar los diferentes pasos a dar y los errores encontrados durante el desarrollo mediante su capacidad
            de crear \textit{issue}. Aparte, Github permite el acceso al código por parte de cualquier persona y ofrece
            la posibilidad de contribuir al proyecto a todo el que quiera.

            El repositorio con el código del proyecto disponible para que cualquier persona pueda verlo está en este
            enlace: https://github.com/jesusjimsa/Drum-It-Yourself.

            Además, el texto en formato LaTeX y las imágenes de esta memoria están disponibles también en este
            repositorio: https://github.com/jesusjimsa/Documentacion-TFG

            \begin{figure}[ht]
                \centering
                \includegraphics[width=\textwidth]{github_issues}
                \caption{Captura de pantalla de las issues de Github del proyecto \label{fig:GithubIssues}}
            \end{figure}

            \newpage

        % subsection Github (end)

        \subsection{Clockify} % (fold)
        \label{sub:Clockify}

            Clockify es una aplicación de seguimiento de tiempo. Ha sido utilizada para llevar un seguimiento de cuánto
            tiempo ha sido dedicado a la creación y desarrollo de este proyecto y su memoria.

            \begin{figure}[ht]
                \centering
                \includegraphics[width=\textwidth/2]{clockify}
                \caption{Captura de pantalla de Clockify \label{fig:ClockifyCaptura}}
            \end{figure}

        % subsection Clockify (end)

        \subsection{Overleaf} % (fold)
        \label{sub:Overleaf}

            Overleaf es un editor de LaTeX online y colaborativo. Permite crear documentos en LaTeX y la colaboración de
            hasta dos personas en su modalidad gratuita.

            En este proyecto, ha facilitado la compartición de esta memoria entre alumno y tutor para su corrección, a
            parte de servir como editor principal de LaTeX.

            \begin{figure}[ht]
                \centering
                \includegraphics[width=\textwidth]{captura_overleaf}
                \caption{Captura de pantalla de Overleaf \label{fig:OverleafCaptura}}
            \end{figure}

        % subsection Overleaf (end)

    % section Tecnologías empleadas (end)

    \section{Descripción a alto nivel} % (fold)
    \label{sec:DescripcionAAltoNivel}

        La solución propuesta consta de dos partes principales. La primera parte es la recogida de información y la
        segunda, el tratamiento de la información y reproducción del sonido.

        \subsection{Recogida de información} % (fold)
        \label{sub:RecogidaDeInformacion}

            En esta primera fase se recoge la información que devuelven los sensores de fuerza en la placa Arduino. Los
            cinco sensores se conectan a la Arduino y devuelven un valor entre 0 y 1023. Este valor se junta a los
            valores leídos por todos los sensores y se manda en un solo mensaje a la Raspberry Pi para empezar la
            segunda fase.

            \newpage

            \begin{figure}[ht]
                \centering
                \includegraphics[width=10cm]{flow_arduino}
                \caption{Diagrama de la recogida de información \label{fig:DiagramaRecogida}}
            \end{figure}

            \begin{figure}[ht]
                \centering
                \includegraphics[width=10cm]{estructura_del_mensaje}
                \caption{Diagrama de la estrutura del mensaje (Valores posibles de 0 a
                1023) \label{fig:DiagramaEstrutura}}
            \end{figure}

        % subsection Recogida de información (end)

        \subsection{Tratamiento de la información y reproducción del sonido} % (fold)
        \label{sub:TratamientoDeLaInformacionYReproduccionDelSonido}

            En esta segunda parte, se recibe la información recogida por la Arduino y se reproducen los sonidos de
            batería correspondientes. La Arduino envía un mensaje por el log del monitor serie y la Raspberry Pi analiza
            este mensaje. Una vez separado el mensaje en pares de intrumento y volumen, se reproducen los sonidos
            correspondientes.

            \begin{figure}[ht]
                \centering
                \includegraphics[width=10cm]{flow_raspberry_pi}
                \caption{Diagrama del tratamiento de la información y la reproducción del
                sonido \label{fig:DiagramaTratamiento}}
            \end{figure}

        % subsection Tratamiento de la información y reproducción del sonido (end)

    % section Descripción a alto nivel (end)

    \section{Decisiones tomadas} % (fold)
    \label{sec:DecisionesTomadas}

        La primera decisión fue entre hacer detección binaria de la entrada, es decir, si el parche ha sido golpeado o
        no, y hacer que estas entradas sean concurrentes (al golpear dos parches, el sonido de ambos suena al mismo
        tiempo), o hacer detección de distintos sonidos en un mismo parche, dependiendo de cómo se golpee el parche (en
        el centro, en el lateral, con más o menos fuerza…) el sonido emitido es diferente.

        Se decide empezar con la primera alternativa y dejar la segunda para más adelante, en caso de tener tiempo.

        Finalmente, por temas de coste, se decide no realizar la detección de distintos en un mismo parche, ya que cada
        sensor de fuerza cuesta 10,19\euro{} por lo que añadir más sensores a cada parche haría que el precio final de
        desarrollar el proyecto se elevara demasiado.

        \subsection{Biblioteca de reproducción de sonido} % (fold)
        \label{sub:LibreriaDeReproduccionDeSonido}

            \subsubsection{playsound} % (fold)
            \label{ssub:Playsound}

                La primera idea de lenguaje de programación para implementar el proyecto fue Python, por ser un lenguaje
                sencillo y con gran variedad de bibliotecas. Al investigar las bibliotecas de reproducción de sonidos
                disponibles para Python, la más recomendada era playsound \cite{playsound}. Se empezaron a hacer pruebas
                con esta librería, pero la salida del sonido era demasiado lenta para un proyecto como este, en el que
                el tiempo que pasa entre que se golpea un parche y se produce la salida del sonido, tiene que ser el
                menor posible. Por esta razón, se decide descartar el uso de Python y playsound.

            % subsubsection playsound (end)

            \subsubsection{mpg123} % (fold)
            \label{ssub:Mpg123}

                Tras decidir no utilizar playsound de Python, se empezó a investigar bibliotecas de otros lenguajes de
                programación. La biblioteca de C, Mpg123 \cite{mpg123}, es una de las más mencionadas en la reproducción
                de sonido. Por esta razón, se hacen pruebas con la biblioteca y se confirma que es lo suficientemente
                rápida para el proyecto, así que se decide seguir adelante con ella y es la biblioteca con la que se ha
                desarrollado el programa que gestiona los sonidos.

                Tiene, sin embargo, un problema en el uso de hebras POSIX para reproducir varios sonidos al mismo
                tiempo. Para solucionar este problema, se utilizan procesos (lanzados por \texttt{fork()}) en lugar de
                hebras POSIX, como se explica en la sección \ref{sub:HebrasVSProcesos}.

            % subsubsection mpg123 (end)

            \subsubsection{libao} % (fold)
            \label{ssub:Libao}

                Para utilizar la bibliteca Mpg123 para la reproducción de sonidos, hay que utilizarla en conjunto con
                libao \cite{libao}. Mpg123 se encarga de descodificar el archivo MP3 que contiene el sonido
                correspondiente y prepararlo para su reproducción, mientras que libao se encarga de mandar la señal al
                sistema operativo para reproducirlo. Utilizando estas dos funciones conjuntamente es como se reproduce
                el sonido en este proyecto.

            % subsubsection libao (end)

        % subsection Biblioteca de reproducción de sonido (end)

        \subsection{Otras bibliotecas} % (fold)
        \label{sub:OtrasLibrerias}

            \subsubsection{wiringPi} % (fold)
            \label{ssub:WiringPi}

                En los primeros prototipos del proyecto se utilizaban botones en lugar de sensores. Estos botones
                estaban directamente conectados a la Raspberry Pi, ya que aún no se contaba con la Arduino. Por este
                motivo apareció la necesidad de encontrar una biblioteca que controlara estos botones y la información
                que le mandaban al programa. La bibliteca más utilizada para este propósito es wiringPi \cite{wiringPi}.
                Siendo incluso recomendada por la propia Raspberry Pi Foundation \cite{wiringpi_raspberrypi_docu}.

                Finalmente, como se explica en la sección \ref{sec:ArduinoVsRaspberryPi}, se decide utilizar una placa
                Arduino para la lectura de los sensores, con lo cual la biblioteca wiringPi ya no es necesaria y no es
                utilizada en la versión final del proyecto.

            % subsubsection wiringPi (end)

        % subsection Otras bibliotecas (end)

        \subsection{if-else vs switch} % (fold)
        \label{sub:if-else_vs_switch}

            Al pulsar una tecla, el número leído se envía a una función que selecciona qué sonido hay que reproducir en
            ese momento, dependiendo de qué sonido corresponda a ese número. Este proceso de selección se puede hacer
            con una estructura de \textit{if-else} anidados o con un \textit{switch-case}.

            Para decidir cuál de las dos soluciones se implementa en la versión final se realizó un test en el que cada
            vez se ejecutan más iteraciones del programa cambiando de sonido en cada una de ellas. Se empieza con 1
            iteración y se termina con 10000000 iteraciones.

            \begin{figure}[ht]
                \centering
                \includegraphics[width=\textwidth]{grafica_if_switch}
                \caption{Gráfica comparativa if-else vs switch \label{fig:GraficaIfVsSwitch}}
            \end{figure}

            \begin{center}
                \begin{tabular}{ |c|c|c| }
                    \hline
                        iterations & if & switch \\
                        \hline\hline
                        1 & 0.000243 & 0.000270 \\
                        \hline
                        10 & 0.002797 & 0.002485 \\
                        \hline
                        100 & 0.027775 & 0.027261 \\
                        \hline
                        1000 & 0.260075 & 0.261464 \\
                        \hline
                        10000 & 0.431544 & 0.425668 \\
                        \hline
                        100000 & 1.368561 & 1.374575 \\
                        \hline
                        1000000 & 8.070825 & 7.560718 \\
                        \hline
                        10000000 & 79.199539 & 71.409653 \\
                    \hline
                \end{tabular}
            \end{center}

            Como se puede ver en la figura \ref{fig:GraficaIfVsSwitch}, la diferencia no es apreciable hasta las 1000000
            iteraciones, pero después pasa a casi 8 segundos de diferencia en 10000000 iteraciones. Por esta razón se ha
            decidido que la función utilice la estructura \textit{switch-case}.

            Finalmente, debido a la manera en la que realizan las comprobaciones de qué botones y sensores son
            utilizados, aunque un \textit{switch-case} es más rápido, esta estructura se reserva para la versión del
            programa que reproduce los sonidos leyendo del teclado. En el programa que controla los sensores se utiliza
            una estructura \textit{if-else}.

        % subsection if-else vs switch (end)

    % section Decisiones tomadas (end)

    \section{Arduino vs Raspberry Pi} % (fold)
    \label{sec:ArduinoVsRaspberryPi}

        Para la lectura de las señales de los sensores de presión RP c18.3 y RP S40, se plantean dos opciones, se puede
        utilizar la propia Raspberry Pi en la que se ejecuta el programa que maneja los sonidos o una Arduino Nano. En
        el proyecto resultante se utiliza finalmente la Arduino debido a dos razones principales.

        La primera razón es el precio y la escalabilidad, una Raspberry Pi cuesta 39,95\euro{} mientras que una Arduino
        Nano cuesta 10\euro{}. Una Arduino Nano cuenta con menos pines de E/S, pero añadir una placa es más barato y
        sencillo que añadir una placa de Raspberry Pi.

        La segunda razón es la implementación del programa que se encarga de el sensor de presión. En Internet se pueden
        encontrar ejemplos y tutoriales refiriéndose a cómo implementar el sistema en una Arduino, pero no es tan fácil
        encontrar información para hacerlo desde una Raspberry Pi.

        Por estas razones se elige realizar la recepción de las señales del sensor de presión desde la Arduino, haciendo
        el proceso más sencillo y más barato.

        \subsection{Conexión} % (fold)
        \label{sub:Conexion}

            Para realizar la conexión de los sensores se utilizan cables de protoboard conectados de la forma explicada
            en la figura \ref{fig:EsquemaConexion}:

            \begin{figure}[ht]
                \centering
                \includegraphics[width=5cm]{force_sensor_arduino}
                \caption{Esquema de conexión de sensores de presión \cite{force_sensor_arduino}
                         \label{fig:EsquemaConexion}}
            \end{figure}

        % subsection Conexión (end)

    % section Arduino vs Raspberry Pi (end)

% chapter Diseño (end)
