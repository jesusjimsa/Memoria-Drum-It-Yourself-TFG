%%%%%%%%%%%%%%%%%%%%%%%%%%%%%%%%%%%%%%%%%%%%%%%%%%%%%%%%%%%%%%%%%%%%%%%%%%%%%%%%%
% Implementación
%%%%%%%%%%%%%%%%%%%%%%%%%%%%%%%%%%%%%%%%%%%%%%%%%%%%%%%%%%%%%%%%%%%%%%%%%%%%%%%%%

\chapter{Implementación de la propuesta}
\label{cha:Implementacion}

    Poner especial atención al proceso: dificultades y problemas encontrados y cómo se han solucionado

    \section{Sonido en paralelo} % (fold)
    \label{sec:SonidoEnParalelo}

        Hasta febrero, todas las versiones del programa fueron diseñadas pensando solo en emitir un sonido cada vez,
        pero lo preferible en un proyecto como este es que se emita más de un sonido al mismo tiempo.\newline

        El primer acercamiento fue mediante un sistema de máximos y sonidos combinados. En esta primera solución, se
        escogía el mayor de los seis sonidos y, si había un segundo sonido con un valor mayor que 200 (el mínimo para
        emitir sonido), se enviaba un mensaje a través del log de Arduino seleccionando un sonido combinado. Estos
        sonidos habían sido combinados previamente utilizando los sonidos ya presentes en el proyecto.\newline

        Finalmente, esta solución no funcionó correctamente (no se enviaba la señal del sonido combinado) y se procedió
        a diseñar una nueva. Esta nueva solución es la que se utiliza actualmente en el proyecto. Consiste en mandar
        todas las señales al mismo tiempo. Antes de la implementación de esta solución, el log era así:

        \begin{verbatim}
        0:0
        0:0
        2:364
        0:0
        0:0
        \end{verbatim}

        Tras la implementación de la solución, el log es así:

        \begin{verbatim}
        0:0
        0:0
        1:446:2:0:3:812:4:0:5:0
        0:0
        0:0
        1:0:2:0:3:0:4:902:5:0
        0:0
        \end{verbatim}

        En cuanto un sensor detecta una presión mayor a 200, se manda el mensaje de todos los sensores al mismo tiempo.
        En caso de que el valor leído sea menor que 200, se manda como 0, pero si es mayor, se manda con los demás. Este
        mensaje es leído y procesado por la Raspberry Pi, que lanzará los procesos necesarios con los sonidos que hagan
        falta según los sensores que hayan sido presionados.\newline

        Con esta segunda propuesta, el programa envía correctamente la señal de todos los sensores y los sonidos son
        emitidos correctamente.

    % section Sonido en paralelo (end)

    \section{Problemas} % (fold)
    \label{sec:Problemas}

        \begin{itemize}
            \item
            Debido a la librería utilizada para reproducir sonidos, al lanzar hebras para reproducir varios sonidos al
            mismo tiempo, se producía en error de \textit{segmentation fault}. La solución a este problema fue sustituir
            las hebras por procesos.
            % \item
            % Relacionado con el problema anterior, intentando reproducir dos sonidos al mismo tiempo, para reducir
            % la posibilidad de que el usuario toque dos instrumentos al mismo tiempo. En el modelo solo por procesos
            % se editan los sonidos juntos para que no haya latencia al tocar dos a la vez, haciéndolo con hebras se
            % ahorra el trabajo de editar los sonidos y no hay combinaciones no contempladas. Sin embargo, el sistema
            % de reproducción de sonido del sistema operativo no permite reproducir sonidos mediante hebras, solo
            % procesos.
            \item
            En el prototipo con botones, al pulsar o dejar pulsado un botón, el programa reproduciría el mismo
            sonido muchas veces. Para solucionar esto se crea una hebra por cada botón, cuando se pulsa, entrará
            en un bucle infinito del que no saldrá hasta que el botón no sea soltado. Al usarse hebras, nos
            permite pulsar más botones al mismo tiempo.
            \item
            El sensor de presión devuelve muchas lecturas por segundo. Para solucionar esto a la hora de reproducir los
            sonidos hay dos formas de solucionarlo: una es introduciendo un \textit{delay} lo suficientemente grande
            para diferenciar dos toques del sensor, la otra solución, que ha sido la implementada, trata de bloquear el
            sensor cada vez que se entra en uno de los tres intervalos de volumen que se han elegido, cada vez que entra
            de deja de leer hasta que no baje la presión lo suficiente. Si la presión sube tampoco enviará señal para
            que reproduzca sonido.
            \item
            Al añadir el sensor y el Arduino, el programa que controlaba los sonidos emitidos recibía las mediciones
            del Arduino y, dependiendo de los datos entregados por éste, se emite un sonido a un volumen concreto.
            La construcción de la cadena de texto que contenía el \textit{path} se hacía mediante las funciones de
            copia y concatenación \textit{strcat} y \textit{strdup}. El problema es que al recibir el \textit{path},
            la biblioteca de reproducción de sonidos lanzaba el siguiente error:

            \begin{verbatim}
            malloc(): corrupted top size
            make: *** [Makefile:19: run] Segmentation fault
            \end{verbatim}

            Tras muchas pruebas, como aumentar la cantidad de memoria reservada para el \textit{path} o para el
            buffer que se utiliza en la función de reproducción, o probar a que siempre se enviara el mismo path,
            sin leer del Arduino (reproduciendo el sonido satisfactoriamente), finalmente decido cambiar la forma en
            la que se genera el path, \textit{hardcodeándolo} en el programa. Esto resulta funcionar y es la solución
            que ha sido implementada en el programa.
        \end{itemize}

    % section Problemas (end)

    \section{Pads} % (fold)
    \label{sec:Pads}

        Los pads son las superficies que son golpeadas para generar los sonidos. Para este proyecto se han fabricado
        usando madera, cola y gomaeva\cite{GomaEva}. El coste total es de fabricar dos pads es de 3,09\euro{}.
        Comparado con otros productos similares como los de Prologix\cite{practice_pad}, cuyo kit de práctica de 4
        pads cuesta \$224,99, el precio de la solución propuesta en este proyecto es sensiblemente inferior.

    % section Pads (end)

    \section{Coste y presupuesto} % (fold)
    \label{sec:CosteYPresupuesto}

        tabla con el coste de desarrollo (incluye tu mano de obra... etc.)

        \subsection{Materiales} % (fold)
        \label{sub:Materiales}

            \begin{itemize}
                \item Dos hojas de goma eva: 1,20\euro{}
                \item Tres tablas madera MDF 600x300x10 mm: 7,47\euro{}
                \item Cables protoboard: 2,60\euro{}
                \item Sensor de fuerza Exing c18.3: 5,91\euro{}
                \item Tres sensores de fuerza Exing RP de S40: 50,95\euro{}
                \item Raspberry Pi 3B: 39,95\euro{}
                \item Tarjeta micro SD 8 GB: 5\euro{}
                \item Caja Aukru + cable alimentación: 11,99\euro{}
                \item Arduino Nano: 15,43\euro{} (incluye gastos de envío)
            \end{itemize}

        % subsection Materiales (end)

    % section Coste y presupuesto (end)


    \section{Tiempo} % (fold)
    \label{sec:Tiempo}

        Para medir el tiempo que ha llevado realizar este trabajo, se ha utilizado la aplicación
        Clockify\cite{clockify}.\newline

        En total se han utilizado 74 horas y 24 minutos.%%%%%%%%%%%%%%%%%%%%%%%%%%%%%%%%%%%%%%%%%%%%%%%%%%%%%%
        %%%%%%%%%%%%%%%%%%%%%%%%%%%%%%%%%%%%%%%%%%%%%%%%%%%%%%%%%%%%%%%%%%%%%%%%%%%%%%%%%%%%%%%%%%%%%%%%%%%%%%
        %%%%%%%%%%%%%%%%%%%%%%%%%%%%%%%%%%%%%%%%%%%%%%%%%%%%%%%%%%%%%%%%%%%%%%%%%%%%%%%%%%%%%%%%%%%%%%%%%%%%%%

    % section Tiempo (end)

% chapter Implementación (end)

\newpage
