%%%%%%%%%%%%%%%%%%%%%%%%%%%%%%%%%%%%%%%%%%%%%%%%%%%%%%%%%%%%%%%%%%%%%%%%%%%%%%%%%
% Conclusiones
%%%%%%%%%%%%%%%%%%%%%%%%%%%%%%%%%%%%%%%%%%%%%%%%%%%%%%%%%%%%%%%%%%%%%%%%%%%%%%%%%

\chapter{Conclusiones}
\label{cha:Conclusiones}

    Llegamos al final del proyecto. Durante los meses que se han invertido en la realización del trabajo, se han
    aprendido varias tecnologías y áreas.

    Se ha aprendido a conectar y programar sensores a placas Raspberry Pi y Arduino. Siguiendo con este tema, se ha
    visto cómo conectar los dos sistemas, para que la Arduino envíe información a la Raspberry Pi para el tratamiento de
    los datos recogidos por los sensores instalados en la Arduino.

    Por otro lado, se ha aprendido a reproducir archivos de sonido en entornos Linux/Unix de manera eficiente y
    simultánea utilizando el lenguaje de programación C.

    Finalmente, se ha alcanzado un resultado que cumple con los objetivos que se marcaron al comienzo del trabajo.
    Utilizando lo aprendido anteriormente, se logra una experiencia de tocar la batería parecida a la realidad,
    incluyendo la construcción de una batería física que poder tocar.

    \section{Trabajo futuro} % (fold)
    \label{sec:TrabajoFuturo}

        Durante la realización de este proyecto, ha surgido una serie de ideas que mejorarían la experiencia de tocar la
        batería, pero que, por falta de tiempo, no ha sido posible desarrollar. Las más destacables son:

        \begin{itemize}
            \item \textbf{Creación de una interfaz web:} Esta interfaz web podría utilizarse para arrancar y pausar la
            batería de forma sencilla, sin necesidad de entrar a la terminal para ello.
            \item \textbf{Sonidos personalizados:} Añadir un apartado a la interfaz web que permitiera al usuario
            añadir sus propios sonidos personalizados para configurar la batería a su gusto.
            \item \textbf{Aprender canciones:} Mediante una serie de LEDs, que se encienden en el momento preciso, se
            añadiría una canción a la batería y el usuario aprendería a tocarla de forma sencilla. Encendiendo el LED
            del pad que haya que tocar y apagándolo cuando haya sido tocado.
            \item \textbf{Guardar canción tocada:} Grabar lo que toque el usuario durante una cantidad limitada de
            tiempo en un archivo de audio para poder reproducirlo más tarde.
        \end{itemize}

    % section Trabajo futuro (end)

% chapter Conclusiones (end)

\newpage
