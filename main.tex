\documentclass{article}
\usepackage[utf8]{inputenc}

\title{Desarrollo de un instrumento musical digital}
\author{ Jesús Jiménez Sánchez }

\begin{document}

\maketitle

\section*{Objetivos}
Objetivo General: 

Objetivos específicos:
\begin{itemize}
    \item OB-E1:
    \item OB-E2:
\end{itemize}

\section{Introducción}
Para empezar se decide entre hacer detección binaria de la entrada, es decir, si el parche ha sido golpeado o no, y hacer que estas entradas sean concurrentes (al golpear dos parches, el sonido de ambos suena al mismo tiempo), o hacer detección de distintos sonidos en un mismo parche, dependiendo de cómo se golpee el parche (en el centro, en el lateral, con más o menos fuerza…) el sonido emitido es diferente.

Se decide empezar con la primera alternativa y dejar la segunda para más adelante en caso de tener tiempo.

Tal y como indica la documentación sobre Arduino \cite{DocArduino}

\subsection{Planificación}

 \subsubsection{A priori (optimista)}

 \subsubsection{A posteriori (en caso de haber diferencias)} 
 
 

%%%%%%%%%%%%%%%%%%%%%%%%%%%%%%%%%%%%%%%%%%%%%%%%%%%%%%%%%%%%%%%%%%%%%%%%%%%%%%%%%
% Diseño
%%%%%%%%%%%%%%%%%%%%%%%%%%%%%%%%%%%%%%%%%%%%%%%%%%%%%%%%%%%%%%%%%%%%%%%%%%%%%%%%%
\section{Diseño de la propuesta}

 Aquí hay que detallar y justificar las decisiones que se tomen (hacer varias propuestas pero desarrollar solo la mejor) (siempre es bueno una tabla comparativa con pros y contras o ticks)
 
 \subsection{Librería de reproducción de sonido ¿?}
 \begin{itemize}
     \item playsound
 \end{itemize} 
 
 
%%%%%%%%%%%%%%%%%%%%%%%%%%%%%%%%%%%%%%%%%%%%%%%%%%%%%%%%%%%%%%%%%%%%%%%%%%%%%%%%%
% Diseño
%%%%%%%%%%%%%%%%%%%%%%%%%%%%%%%%%%%%%%%%%%%%%%%%%%%%%%%%%%%%%%%%%%%%%%%%%%%%%%%%%
\section{Implmentación de la propuesta}

 Poner especial atención al proceso: dificultades y problemas encontrados y cómo se han solucionado
 
 
 \subsection{Coste y presupuesto}
  
  tabla con el coste de desarrollo (incluye tu mano de obra... etc.)
  
->  Identifica los aspectos éticos y sociales relacionados con la profesión

 
 %%%%%%%%%%%%%%%%%%%%%%%%%%%%%%%%%%%%%%%%%%%%%%%%%%%%%%%%%%%%%%%%%%%%%%%%%%%%%%%%%
% Resultados y discusión
%%%%%%%%%%%%%%%%%%%%%%%%%%%%%%%%%%%%%%%%%%%%%%%%%%%%%%%%%%%%%%%%%%%%%%%%%%%%%%%%%
\section{Resultados y discusión}\label{sec:ResultadosDisc}

-> Identifica los aspectos éticos y sociales relacionados con la profesión

%%%%%%%%%%%%%%%%%%%%%%%%%%%%%%%%%%%%%%%%%%%%%%%%%%%%%%%%%%%%%%%%%%%%%%%%%%%%%%%%%
% Conclusiones
%%%%%%%%%%%%%%%%%%%%%%%%%%%%%%%%%%%%%%%%%%%%%%%%%%%%%%%%%%%%%%%%%%%%%%%%%%%%%%%%%
\section{Conclusiones}

 ¿se han satisfecho los objetivos? ¿el porqué? Como se ha vbisto en \ref{sec:ResultadosDisc}, los o...
 
-> Identifica los aspectos éticos y sociales relacionados con la profesión



\bibliographystyle{plain}
\bibliography{referencias}

\end{document}

