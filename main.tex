\documentclass{article}
\usepackage[utf8]{inputenc}

\title{Desarrollo de un instrumento musical digital}
\author{ Jesús Jiménez Sánchez }

\begin{document}

\maketitle

%%%%%%%%%%%%%%%%%%%%%%%%%%%%%%%%%%%%%%%%%%%%%%%%%%%%%%%%%%%%%%%%%%%%%%%%%%%%%%%%%
% Objetivos
%%%%%%%%%%%%%%%%%%%%%%%%%%%%%%%%%%%%%%%%%%%%%%%%%%%%%%%%%%%%%%%%%%%%%%%%%%%%%%%%%
\section*{Objetivos}\label{sec:Objetivos}
Objetivo General:

Objetivos específicos:
\begin{itemize}
    \item OB-E1:
    \item OB-E2:
\end{itemize}

%%%%%%%%%%%%%%%%%%%%%%%%%%%%%%%%%%%%%%%%%%%%%%%%%%%%%%%%%%%%%%%%%%%%%%%%%%%%%%%%%
% Introducción
%%%%%%%%%%%%%%%%%%%%%%%%%%%%%%%%%%%%%%%%%%%%%%%%%%%%%%%%%%%%%%%%%%%%%%%%%%%%%%%%%
\section{Introducción}\label{sec:Introduccion}
Para empezar se decide entre hacer detección binaria de la entrada, es decir, si el parche ha sido golpeado o no, y hacer que estas entradas sean concurrentes (al golpear dos parches, el sonido de ambos suena al mismo tiempo), o hacer detección de distintos sonidos en un mismo parche, dependiendo de cómo se golpee el parche (en el centro, en el lateral, con más o menos fuerza…) el sonido emitido es diferente.

Se decide empezar con la primera alternativa y dejar la segunda para más adelante en caso de tener tiempo.

Tal y como indica la documentación sobre Arduino \cite{DocArduino}

\subsection{Derechos de autor}
\subsubsection{¿Qué son los derechos de autor?}
Los derechos de autor son una serie de leyes que protegen la autoría de las obras. Estas pueden ser libros, películas, obras de teatro, programas informáticos...\newline
Se cubren dos tipos de derechos: los derechos patrimoniales, que aseguran que el autor obtenga compensación financiera, y los derechos morales, que cubren todo lo que no esté relacionado con los derechos patrimoniales, por ejemplo, la prohibición de que se modifique la obra.\cite{derechos_ompi}

\subsubsection{Historia de los derechos de autor}

\subsection{Planificación}

 \subsubsection{A priori (optimista)}

 \begin{itemize}
    \item Programa que genere los sonidos de batería concurrentes: 1-2 meses
    \item Construcción de un prototipo de batería con sonidos concurrentes: 1-2 meses
    \item Añadir funcionalidad que genere diferentes sonidos de un mismo parche/platillo: 1 mes
    %% El programa tiene que diferenciar el lugar en el que se golpea el parche/platillo, lo que será la mayor dificultad
    \item Añadir sensores y conexiones para generar diferentes sonidos según dónde se golpee en el parche/platillo: 1-2 semanas
    \item Aplicación web para conectar dos baterías online: 1-2 meses
    \item Documentación: 3 semanas
\end{itemize}

 \subsubsection{A posteriori (en caso de haber diferencias)}
%%%%%%%%%%%%%%%%%%%%%%%%%%%%%%%%%%%%%%%%%%%%%%%%%%%%%%%%%%%%%%%%%%%%%%%%%%%%%%%%%
% Diseño
%%%%%%%%%%%%%%%%%%%%%%%%%%%%%%%%%%%%%%%%%%%%%%%%%%%%%%%%%%%%%%%%%%%%%%%%%%%%%%%%%
\section{Diseño de la propuesta}\label{sec:Diseno}

 Aquí hay que detallar y justificar las decisiones que se tomen (hacer varias propuestas pero desarrollar solo la mejor) (siempre es bueno una tabla comparativa con pros y contras o ticks)

 \subsection{Librería de reproducción de sonido ¿?}
 \begin{itemize}
     \item playsound\cite{playsound}: No se usa porque es muy lento y el programa que queremos crear necesita ser lo más rápido posible.
     \item mpg123\cite{mpg123}: Librería y programa en C más rápido que playsound de Python. Tiene el problema de hacer que haya fallos de memoria cuando se usan hebras para reproducir varios sonidos al mismo tiempo.
 \end{itemize}

 \subsection{Problemas ¿?}
 \begin{itemize}
     \item Error de \textit{segmentation fault} que ocurría por utilizar hebras para reproducir varios sonidos al
     mismo tiempo, se solucionó cambiando las hebras por procesos.
     \item Relacionado con el problema anterior, intentando reproducir dos sonidos al mismo tiempo,
        para reducir la posibilidad de que el usuario toque dos intrumentos al mismo tiempo
 \end{itemize}


%%%%%%%%%%%%%%%%%%%%%%%%%%%%%%%%%%%%%%%%%%%%%%%%%%%%%%%%%%%%%%%%%%%%%%%%%%%%%%%%%
% Implementación
%%%%%%%%%%%%%%%%%%%%%%%%%%%%%%%%%%%%%%%%%%%%%%%%%%%%%%%%%%%%%%%%%%%%%%%%%%%%%%%%%
\section{Implementación de la propuesta}\label{sec:Implementacion}

 Poner especial atención al proceso: dificultades y problemas encontrados y cómo se han solucionado

 \subsection{Coste y presupuesto}

  tabla con el coste de desarrollo (incluye tu mano de obra... etc.)

->  Identifica los aspectos éticos y sociales relacionados con la profesión

 %%%%%%%%%%%%%%%%%%%%%%%%%%%%%%%%%%%%%%%%%%%%%%%%%%%%%%%%%%%%%%%%%%%%%%%%%%%%%%%%%
% Resultados y discusión
%%%%%%%%%%%%%%%%%%%%%%%%%%%%%%%%%%%%%%%%%%%%%%%%%%%%%%%%%%%%%%%%%%%%%%%%%%%%%%%%%
\section{Resultados y discusión}\label{sec:ResultadosDisc}

-> Identifica los aspectos éticos y sociales relacionados con la profesión


%%%%%%%%%%%%%%%%%%%%%%%%%%%%%%%%%%%%%%%%%%%%%%%%%%%%%%%%%%%%%%%%%%%%%%%%%%%%%%%%%
% Conclusiones
%%%%%%%%%%%%%%%%%%%%%%%%%%%%%%%%%%%%%%%%%%%%%%%%%%%%%%%%%%%%%%%%%%%%%%%%%%%%%%%%%
\section{Conclusiones}\label{sec:Conclusiones}

 ¿se han satisfecho los objetivos? ¿el porqué? Como se ha visto en \ref{sec:ResultadosDisc}, los o...

-> Identifica los aspectos éticos y sociales relacionados con la profesión


\bibliographystyle{plain}
\bibliography{referencias}

\end{document}

