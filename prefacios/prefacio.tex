%%%%%%%%%%%%%%%%%%%%%%%%%%%%%%%%%%%%%%%%%%%%%%%%%%%%%%%%%%%%%%%%%%%%%%%%%%%%%%%%%
% Prefacio
%%%%%%%%%%%%%%%%%%%%%%%%%%%%%%%%%%%%%%%%%%%%%%%%%%%%%%%%%%%%%%%%%%%%%%%%%%%%%%%%%

\chapter*{}   %%%%%%%%%%%%%%%%%%%%%%

\cleardoublepage
\thispagestyle{empty}

\begin{center}
    {\large\bfseries Desarrollo de un instrumento musical digital}\\
\end{center}

\begin{center}
    Jesús Jiménez Sánchez\\
\end{center}

%\vspace{0.7cm}
\noindent{\textbf{Palabras clave}: batería, sonido, Raspberry Pi, Arduino, procesos, sensor}\\

\vspace{0.7cm}
\noindent{\textbf{Resumen}}\\

Este proyecto consiste en el desarrollo una batería eléctrica utilizando una serie de sensores conectados a las placas
Raspberry Pi y Arduino.

Los sensores están conectados a la placa Arduino, que recibe los valores leídos por los sensores de fuerza, y genera un
mensaje que mandará a la Raspberry Pi.

En la Raspberry Pi, se leen los mensajes enviados por la Arduino y se decodifican para seleccionar un archivo de audio
que se reproducirá por un altavoz.

Todo esto está colocado en una estructura que simula una batería.

En esta documentación se hablará sobre las placas Raspberry Pi y Arduino y sus modelos, hardware y alternativas. También
se tratará el tema de los derechos de autor, su historia, la ley española y cuál es la protección de los archivos de
audio utilizados en el proyecto.

Por último, se verá cómo se ha hecho el diseño e implementación del proyecto, comentando las decisiones tomadas, el
funcionamiento del programa y los problemas encontrados en el desarrollo de la idea.

\cleardoublepage

\thispagestyle{empty}

\begin{center}
       {\large\bfseries Development of a digital musical instrument}\\
\end{center}

\begin{center}
    Jesús Jiménez Sánchez\\
\end{center}

\noindent{\textbf{Keywords}: drums, sound, Raspberry Pi, Arduino, process, sensor}\\

\vspace{0.7cm}
\noindent{\textbf{Abstract}}\\

This project is about the development of an electric drumset using sensors connected to the Raspberry Pi and Arduino
boards.

The sensors are connected to the Arduino board, which receives the values read by the force sensors, and generates a
message that will be sent to the Raspberry Pi.

In the Raspberry Pi, the messages are read and parsed to select an audio file to play in a speaker.

This is all set in an structure to simulate a drumset.

In this documentation I will talk about the Raspberry Pi and Arduino boards and its models, hardware and alternatives. I
will also discuss the subject of copyright, its history, the Spanish law and what is the protection of the audio files
used in this project.

Finally, we will see how the design and implementation of the project has been done, talking about the decissions made,
how the program works and the problems found during the development of the idea.

\chapter*{}   %%%%%%%%%%%%%%%%%%%%%%

\thispagestyle{empty}

\noindent\rule[-1ex]{\textwidth}{2pt}\\[4.5ex]

Yo, \textbf{Jesús Jiménez Sánchez}, alumno de la titulación \textbf{Grado en Ingeniería Informática} de la
\textbf{Escuela Técnica Superior de Ingenierías Informática y de Telecomunicación de la Universidad de Granada}, con DNI
11111111A, autorizo la ubicación de la siguiente copia de mi Trabajo Fin de Grado en la biblioteca del centro para que
pueda ser consultada por las personas que lo deseen.

\vspace{6cm}

\noindent Fdo: Jesús Jiménez Sánchez

\vspace{2cm}

\begin{flushright}
    Granada a 5 de julio de 2020.
\end{flushright}

\chapter*{}   %%%%%%%%%%%%%%%%%%%%%%
\thispagestyle{empty}

\noindent\rule[-1ex]{\textwidth}{2pt}\\[4.5ex]

D. \textbf{Alberto Guillén Perales}, Profesor del Área de Arquitectura y Tecnología de Computadores del Departamento
Arquitectura y Tecnología de Computadores de la Universidad de Granada.

\vspace{0.5cm}

\textbf{Informan:}

\vspace{0.5cm}

Que el presente trabajo, titulado \textit{\textbf{Desarrollo de un instrumento musical digital}}, ha sido realizado bajo
su supervisión por \textbf{Jesús Jiménez Sánchez}, y autorizamos la defensa de dicho trabajo ante el tribunal que
corresponda.

\vspace{0.5cm}

Y para que conste, expiden y firman el presente informe en Granada a 5 de julio de 2020.

\vspace{1cm}

\textbf{Los directores:}

\vspace{5cm}

\noindent \textbf{Alberto Guillén Perales}

\chapter*{Agradecimientos}
\thispagestyle{empty}

    \vspace{1cm}

A mi tutor, Alberto Guillén, por su ayuda en el desarrollo de este proyecto.

A mi tío José, por ayudarme con la madera de la batería.

A todos los profesores que me han dado clase hasta llegar a este momento. Gracias a su dedicación he conseguido llegar
hasta aquí.

A mi madre Malena, mi padre Luis, mi hermana Ana Isabel y toda mi familia, por estar en todo momento y apoyarme siempre
que lo necesitara.

Y a mi pareja Eva y todos mis amigos, por aguantarme en todos los momentos, los buenos y los malos, de este largo y
sinuoso camino.
